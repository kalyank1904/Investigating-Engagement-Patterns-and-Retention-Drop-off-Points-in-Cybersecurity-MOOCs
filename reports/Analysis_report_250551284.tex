% Options for packages loaded elsewhere
\PassOptionsToPackage{unicode}{hyperref}
\PassOptionsToPackage{hyphens}{url}
\documentclass[
]{article}
\usepackage{xcolor}
\usepackage[margin=1in]{geometry}
\usepackage{amsmath,amssymb}
\setcounter{secnumdepth}{-\maxdimen} % remove section numbering
\usepackage{iftex}
\ifPDFTeX
  \usepackage[T1]{fontenc}
  \usepackage[utf8]{inputenc}
  \usepackage{textcomp} % provide euro and other symbols
\else % if luatex or xetex
  \usepackage{unicode-math} % this also loads fontspec
  \defaultfontfeatures{Scale=MatchLowercase}
  \defaultfontfeatures[\rmfamily]{Ligatures=TeX,Scale=1}
\fi
\usepackage{lmodern}
\ifPDFTeX\else
  % xetex/luatex font selection
\fi
% Use upquote if available, for straight quotes in verbatim environments
\IfFileExists{upquote.sty}{\usepackage{upquote}}{}
\IfFileExists{microtype.sty}{% use microtype if available
  \usepackage[]{microtype}
  \UseMicrotypeSet[protrusion]{basicmath} % disable protrusion for tt fonts
}{}
\makeatletter
\@ifundefined{KOMAClassName}{% if non-KOMA class
  \IfFileExists{parskip.sty}{%
    \usepackage{parskip}
  }{% else
    \setlength{\parindent}{0pt}
    \setlength{\parskip}{6pt plus 2pt minus 1pt}}
}{% if KOMA class
  \KOMAoptions{parskip=half}}
\makeatother
\usepackage{graphicx}
\makeatletter
\newsavebox\pandoc@box
\newcommand*\pandocbounded[1]{% scales image to fit in text height/width
  \sbox\pandoc@box{#1}%
  \Gscale@div\@tempa{\textheight}{\dimexpr\ht\pandoc@box+\dp\pandoc@box\relax}%
  \Gscale@div\@tempb{\linewidth}{\wd\pandoc@box}%
  \ifdim\@tempb\p@<\@tempa\p@\let\@tempa\@tempb\fi% select the smaller of both
  \ifdim\@tempa\p@<\p@\scalebox{\@tempa}{\usebox\pandoc@box}%
  \else\usebox{\pandoc@box}%
  \fi%
}
% Set default figure placement to htbp
\def\fps@figure{htbp}
\makeatother
\setlength{\emergencystretch}{3em} % prevent overfull lines
\providecommand{\tightlist}{%
  \setlength{\itemsep}{0pt}\setlength{\parskip}{0pt}}
\usepackage{bookmark}
\IfFileExists{xurl.sty}{\usepackage{xurl}}{} % add URL line breaks if available
\urlstyle{same}
\hypersetup{
  pdftitle={Learning Analytics of a Cyber Security MOOC},
  pdfauthor={Kalyan Kankanala},
  hidelinks,
  pdfcreator={LaTeX via pandoc}}

\title{Learning Analytics of a Cyber Security MOOC}
\author{Kalyan Kankanala}
\date{2026-01-15}

\begin{document}
\maketitle

{
\setcounter{tocdepth}{2}
\tableofcontents
}
Student ID: 250551284 Module: MAS8505

\begin{enumerate}
\def\labelenumi{\arabic{enumi}.}
\tightlist
\item
  Introduction
\end{enumerate}

Online learning platforms generate large volumes of learner interaction
data, which can be analysed to understand learner behaviour, engagement,
and course outcomes.

This project focuses on learning analytics for a Cyber Security Massive
Open Online Course (MOOC). The primary objective is to analyse learner
enrolment patterns, engagement with course steps, survey responses, and
completion behaviour across multiple course runs.

The key business problem addressed in this analysis is: How learner
engagement influences course completion, and whether engagement metrics
can be used to predict successful course completion.

This analysis aims to provide actionable insights that can help course
designers improve learner retention and engagement.

knitr::include\_graphics(``graphs/completion\_rate\_by\_engagement\_level.png'')

Data Description

The analysis uses datasets from four course runs: Run 1, Run 3, Run 5,
and Run 7.

The following datasets were used:

Enrolment data: learner registration and completion status

Step activity data: learner interactions with course steps

Leaving survey responses: reasons for learners leaving the course

Each dataset contains anonymised learner identifiers and was analysed at
both run-level and learner-level granularity.

Data Cleaning and Preparation

Data cleaning was performed separately for each dataset and run to
ensure consistency.

Key cleaning steps included:

Converting date-time fields to standard formats

Creating binary indicators for step visits and step completion

Handling missing values explicitly

Standardising learner identifiers across datasets

Cleaned datasets were then combined across runs where appropriate to
enable comparative and integrated analysis.

Exploratory Data Analysis

Exploratory analysis was conducted to understand enrolment patterns and
learner demographics.

Key EDA outputs include:

Completion rates by course run

Distribution of learners by country

Missingness analysis for demographic variables

The results indicate substantial variation in completion rates across
runs, with overall completion remaining low --- a common trend in MOOCs.

knitr::include\_graphics(``../graphs/enrolments\_completion\_rate\_by\_run.png'')

\begin{enumerate}
\def\labelenumi{\arabic{enumi}.}
\setcounter{enumi}{4}
\tightlist
\item
  Learner Engagement and Completion
\end{enumerate}

Learner engagement was measured using:

Number of steps visited

Number of steps completed

These engagement metrics were merged with enrolment completion status to
assess how engagement differs between learners who completed the course
and those who did not.

Results show that learners who completed the course had significantly
higher levels of engagement compared to non-completers.

knitr::include\_graphics(``../graphs/enrolments\_top10\_countries.png'')

knitr::include\_graphics(``../graphs/steps\_completed\_by\_completion.png'')

knitr::include\_graphics(``../graphs/steps\_visited\_by\_completion.png'')

\begin{enumerate}
\def\labelenumi{\arabic{enumi}.}
\setcounter{enumi}{5}
\tightlist
\item
  Statistical Analysis
\end{enumerate}

To formally test differences in engagement between completers and
non-completers, non-parametric Wilcoxon rank-sum tests were applied.

The results show statistically significant differences in both:

Steps visited

Steps completed

knitr::kable(
read.csv(``../reports/output/engagement\_summary\_by\_completion.csv''),
caption = ``Engagement statistics by course completion status'' )

This confirms that engagement levels are strongly associated with course
completion.

\begin{enumerate}
\def\labelenumi{\arabic{enumi}.}
\setcounter{enumi}{6}
\tightlist
\item
  Predictive Modelling
\end{enumerate}

Logistic regression models were developed to predict course completion
based on:

Steps completed

Steps visited

Course run

The model demonstrates that the number of steps completed is a strong
positive predictor of course completion, even after accounting for
run-level effects. knitr::kable(
read.csv(``../reports/output/logit\_odds\_ratios.csv''), caption =
``Logistic regression odds ratios for course completion'' )

knitr::include\_graphics(``../graphs/logit\_predicted\_prob\_by\_steps\_completed.png'')

\begin{enumerate}
\def\labelenumi{\arabic{enumi}.}
\setcounter{enumi}{7}
\tightlist
\item
  Key Insights and Business Implications
\end{enumerate}

Key findings from the analysis include:

Learners with low or no engagement have extremely low completion rates

High engagement learners show substantially higher completion
probabilities

Early engagement appears critical for course success

From a business perspective, these insights suggest that:

Early intervention strategies should target low-engagement learners

Course designers should prioritise interactive steps early in the course

Engagement metrics can be used for early warning systems to improve
retention

knitr::include\_graphics(``../graphs/roc\_like\_curve.png'')

\begin{enumerate}
\def\labelenumi{\arabic{enumi}.}
\setcounter{enumi}{8}
\tightlist
\item
  Limitations
\end{enumerate}

This analysis has several limitations:

Engagement was measured only through step interactions

Time-spent data and video analytics were not available

Survey responses were limited to certain course runs

Future analyses could incorporate richer behavioural data and
longitudinal tracking.

\begin{enumerate}
\def\labelenumi{\arabic{enumi}.}
\setcounter{enumi}{9}
\tightlist
\item
  Conclusion
\end{enumerate}

This project demonstrates the value of learning analytics in
understanding learner behaviour in online courses.

The findings show a strong relationship between learner engagement and
course completion, and highlight opportunities for data-driven
interventions to improve learner outcomes.

Overall, this analysis provides meaningful insights that can inform
course design and learner support strategies in MOOCs.

\end{document}
